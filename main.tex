 \documentclass[12pt]{article}
\renewcommand{\baselinestretch}{1.1}

%%% Add packages here
\usepackage{minted}
\usepackage{listings}
\usepackage{xcolor} % for syntax colors
\usepackage{graphicx} 
%\usepackage{parskip}
%\usepackage[table]{xcolor}
\usepackage{amsmath}
%\usepackage{float}
\usepackage{placeins} % prevent float
\usepackage{geometry}
\geometry{legalpaper, portrait, margin=3cm}
\usepackage{makecell}
\usepackage{multirow}
\usepackage{array}
\usepackage{amsfonts}
\usepackage{amsthm}
\usepackage{amssymb}
\usepackage{latexsym}
\usepackage{color}
\usepackage{verbatim}
\usepackage{fancyhdr}
\usepackage{fancybox}
\usepackage{listings}
\usepackage{threeparttable} % for table notes
\usepackage{hologo} % for TeX engine logos
\usepackage{booktabs} % for nice tables
\usepackage{longtable} % for longer tables
%\usepackage[table,xcdraw]{xcolor}  % for color in tables
\usepackage{appendix} % for appendix
\usepackage{biblatex}
\addbibresource{references.bib}

\usepackage{amsmath} % for pseudo code
\usepackage{algorithm}
\usepackage{algpseudocode}
\usepackage{subcaption} % For subfigures
\usepackage{threeparttable} % for table notes

\definecolor{codegreen}{rgb}{0,0.6,0}
\definecolor{codegray}{rgb}{0.5,0.5,0.5}
\definecolor{codepurple}{rgb}{0.58,0,0.82}
\definecolor{backcolour}{rgb}{0.95,0.95,0.92}

\lstdefinestyle{mystyle}{
    backgroundcolor=\color{backcolour},   
    commentstyle=\color{codegreen},
    keywordstyle=\color{magenta},
    numberstyle=\tiny\color{codegray},
    stringstyle=\color{codepurple},
    basicstyle=\ttfamily\footnotesize,
    breakatwhitespace=false,         
    breaklines=true,                 
    captionpos=b,                    
    keepspaces=true,                 
    numbers=left,                    
    numbersep=5pt,                  
    showspaces=false,                
    showstringspaces=false,
    showtabs=false,                  
    tabsize=2
}

\lstset{style=mystyle}

%%% Margins
\addtolength{\oddsidemargin}{-.50in}
\addtolength{\evensidemargin}{-.50in}
\addtolength{\textwidth}{1.0in}
\addtolength{\topmargin}{-.40in}
\addtolength{\textheight}{0.80in}

%%% Header
\pagestyle{fancy}
\chead{\groupname} 
\rhead{}
\lhead{}
\cfoot{\thepage}
\renewcommand{\headrulewidth}{0.4pt}

%%%%%%%%%%%%%%%%%%%%%%%%%%%%%%%%%%%%%
\newcommand{\groupname}{Group 4 - Geostatistics}
\setcounter{secnumdepth}{0}
\begin{document}
\clearpage\thispagestyle{empty}

\begin{titlepage}

\begin{center}
\vspace*{1in}
	% title
	\centering\textbf{\huge{Spatial Epidemiology\\[3cm] }}
	% details
	\Large{\textbf{
	Master of Statistics and Data Science} \\
	Hasselt University\\	
        2024-2025 \\
	}


\vspace*{3cm}
\textbf{\large{Group 4 :}}\\
Winnie Kulei (2469362) \\
Edward Otieno Owinoh (2365191) \\
Loise Kanini (2262458)



\vspace*{2cm}


\vspace*{1.5cm}
\textbf{\large{Lecturers:}}\\
Prof. Dr. Christel Faes \\
Prof. Dr. Thomas Neyens \\

\vspace*{2\baselineskip}
\today

\begin{figure}[b]
   \centering
   \includegraphics[width=8cm]{UHasselt_logog.png}
   \label{fig:Uhasseltlogo}
\end{figure}

\end{center}

\end{titlepage}


%%%%%%%%%%%%%%%%%%%%%%%%%%%%%%%%%%%%%%%%%%%%%%%%%%
\newpage
\section{Background}
Dengue is a mosquito-transmitted viral infection endemic to tropical and subtropical regions where climate conditions support large vector populations. The disease often presents asymptomatically, yet when symptoms occur, they typically include high fever, headache, and intense bodily pain\cite{who_dengue_severe_2025}. Although recovery usually occurs within a couple of weeks, the risk of developing a severe, life-threatening form of the disease underscores its gravity. Since no specific treatment exists, prevention, primarily through avoiding daytime mosquito bites, is the single most vital strategy for reducing the burden of infection \cite{who_dengue_severe_2025}. \\\\
\noindent One way to quantify the local risk of dengue transmission is through the Vector Occurrence (VO) index, a continuous measure that reflects the combined abundance of mosquito vector species at a given location. In entomology, vector occurrence refers to the presence, distribution, and population level of arthropods capable of transmitting pathogens to humans, animals, or plants. The VO index is derived from data collected using both light traps and sentinel traps, providing a mixed and comprehensive estimate of mosquito activity. Higher VO values indicate greater overall mosquito abundance and therefore a higher potential risk of dengue transmission. \\\\
\noindent  The majority of the at-risk population currently lives in the Asia-Pacific Region \cite{who_dengue_severe_2025}. Cambodia has endured a long-term public health challenge from dengue fever, traceable to a massive epidemic in 1995 that resulted in more than 400 recorded deaths. Since this benchmark event, the annual monitoring of dengue incidence reveals a sustained upward trend in case numbers \cite{huy2010national} \cite{vong2010dengue} \cite{herbreteau2025spatio}. This escalating threat has been marked by several significant outbreaks characterized by high morbidity, including major epidemics in 2007 (39,618 cases and 396 deaths), 2012 (42,362 cases and 189 deaths), and most recently in 2019, which registered the highest case count yet at 68,597, alongside 48 fatalities\cite{maquart2021recent}. \\ \\
\noindent This project investigates the spatial variation of the VO index across Cambodia. Because mosquito distribution and abundance are affected by biophysical conditions, the VO index is expected to vary
with environmental factors. The study uses a dataset containing HVI values assessed in a random sample of 214 locations in Colombia and the variables in the dataset are as described in  table \ref{tab:vars} below:


\begin{center}
\begin{tabular}{ll}
\hline
\textbf{Variable} & \textbf{Description} \\
\hline
$VO_i$            & Vector Occurrence Index at location $i$ \\[2pt]
$elevation_i$     & Elevation at location $i$ (meters above sea level) \\
$X_i$             & Longitude of location $i$ \\[2pt]
$Y_i$             & Latitude of location $i$ \\
\hline
\end{tabular}
\label{tab:vars}
\end{center}

\section{Study Objectives}
The objectives of this study were as follows:
\begin{enumerate}
    \item To  determine whether VO exhibits spatial correlation across Cambodia, indicating that neighboring areas may share similar underlying patterns.
    \item To assess whether the observed variation in VO can be explained solely by elevation or whether additional, unobserved spatially structured factors are influencing VO at finer spatial scales.
    \item To develop a model-based map of predicted mean VO across the country, incorporating elevation and any relevant latent spatial components identified in the analysis
    \item  Finally, to support the Cambodian government’s efforts to better characterize high-risk areas—defined as those with VO values of 6 or higher, we design an adaptive sampling strategy and identify 10 new locations where additional data collection would most effectively improve understanding of VO risk.
\end{enumerate}

\section{Methodology}
\subsection{Exploratory Analysis}
All analyses were conducted in R (version 4.5.2), and the code can be found in the Appendix \cite{r_cite}. A map of Cambodia was obtained using the \texttt{world} dataset from the sf package\cite{sf_book}\cite{sf_package}. For both the map and the data, UTM coordinates were first created. For the model predictions, elevation points for all locations in Cambodia were obtained using the \texttt{get\_elev\_point()} function from the \texttt{elevatr} package, using latitude and longitude coordinates \cite{elev}. All other analyses, including the maps of predicted VO Index values, were conducted using UTM coordinates. \\ 
The data were first explored by computing summary statistics for the VO Index and elevation. A map showing the locations of data collection was produced to assess how the data were distributed across Cambodia. Next, a plot of the VO Index against elevation was created to examine the relationship between the two variables. Maps of the VO Index and elevation at the data collection locations were also produced to further explore their association and to visualize any spatial trends across Cambodia.

\subsection{Statistical Analysis}

Summary statistics

% latex table generated in R 4.5.2 by xtable 1.8-4 package
% Mon Dec  8 12:47:46 2025
\begin{table}[ht]
\centering
\begin{tabular}{rrrrrrr}
  \hline
 & Min. & 1st Qu. & Median & Mean & 3rd Qu. & Max. \\ 
  \hline
VO & 0.00 & 5.60 & 6.11 & 5.92 & 6.60 & 8.13 \\ 
  elevation & -10.00 & 18.50 & 72.00 & 132.08 & 130.00 & 1347.00 \\ 
   \hline
\end{tabular}
\end{table}
\begin{figure}[htbp]
    \centering
    \includegraphics[width=0.75\linewidth]{plots/eda_VO.png}
    \caption{VO }
    \label{fig:vo}
\end{figure}

\begin{figure}[htbp]
    \centering
    \includegraphics[width=0.75\linewidth]{plots/eda_elevation.png}
    \caption{Elevation }
    \label{fig:el}
\end{figure}

\begin{figure}[htbp]
    \centering
    \includegraphics[width=0.75\linewidth]{plots/eda_VO_vs_elevation.png}
    \caption{Elevation vs VO }
    \label{fig:el_vo}
\end{figure}
\begin{verbatim}
    Call:
lm(formula = VO ~ elevation, data = VO_Cambodia_Updated)

Residuals:
     Min       1Q   Median       3Q      Max 
-1.29794 -0.41959 -0.07486  0.49279  1.64685 

Coefficients:
              Estimate Std. Error t value Pr(>|t|)    
(Intercept)  6.5549415  0.0488785  134.11   <2e-16 ***
elevation   -0.0047860  0.0002012  -23.79   <2e-16 ***
---
Signif. codes:  0 ‘***’ 0.001 ‘**’ 0.01 ‘*’ 0.05 ‘.’ 0.1 ‘ ’ 1

Residual standard error: 0.6001 on 212 degrees of freedom
Multiple R-squared:  0.7274,	Adjusted R-squared:  0.7262 
F-statistic: 565.8 on 1 and 212 DF,  p-value: < 2.2e-16
\end{verbatim}


\begin{figure}[htbp]
    \centering
    \includegraphics[width=0.75\linewidth]{plots/eda_VO_vs_elevation.png}
    \caption{Elevation vs VO }
    \label{fig:el_vo}
\end{figure}

\begin{figure}
    \centering
    \includegraphics[width=0.75\linewidth]{plots/spatial_correlation_diagnostic.png}
    \caption{Variogram}
    \label{fig:lm_variogram}
\end{figure}


\begin{verbatim}
    Least square fit to the empirical variogram 
sigma^2 = 0.1175444 (Variance of the Gaussian process) 
phi = 68.91269 (Scale of the spatial correlation) 
tau^2 = 0.06385818 (Variance of the nugget effect) 
\end{verbatim}

\begin{figure}
    \centering
    \includegraphics[width=0.9\linewidth]{plots/lm_diag_plots.png}
    \caption{Linear Model Diagnostic Plots}
    \label{fig:lm_Plots}
\end{figure}


Linear geostatistical Model
\begin{verbatim}
                   Estimate      StdErr z.value   p.value    
(Intercept)  6.54123347  0.10395944  62.921 < 2.2e-16 ***
elevation   -0.00468364  0.00022119 -21.174 < 2.2e-16 ***
---
Signif. codes:  0 ‘***’ 0.001 ‘**’ 0.01 ‘*’ 0.05 ‘.’ 0.1 ‘ ’ 1

Log-likelihood: 23.61158
 
Covariance parameters Matern function (kappa=0.5) 
             Estimate StdErr
log(sigma^2)  -1.6873 0.2964
log(phi)       4.5827 0.4412
log(tau^2)    -1.7643 0.7568
\end{verbatim}
\begin{figure}
    \centering
    \includegraphics[width=0.75\linewidth]{plots/geo_spatial_correlation_diagnostic.png}
    \caption{Geostatitical Variogram}
    \label{fig:geo_variogram}
\end{figure}



\begin{verbatim}
    Geostatistical linear model 
Call: 
linear.model.MLE(formula = VO ~ elevation, coords = ~utm_x + 
    utm_y, data = VO_Cambodia_Updated, ID.coords = NULL, kappa = 0.5, 
    fixed.rel.nugget = NULL, start.cov.pars = c(spatial_cor[["lse.variogram"]][["phi"]], 
        spatial_cor[["lse.variogram"]][["tau^2"]]/spatial_cor[["lse.variogram"]][["sigma^2"]]), 
    method = "nlminb")

               Estimate      StdErr z.value   p.value    
(Intercept)  6.54123347  0.10395944  62.921 < 2.2e-16 ***
elevation   -0.00468364  0.00022119 -21.174 < 2.2e-16 ***
---
Signif. codes:  0 ‘***’ 0.001 ‘**’ 0.01 ‘*’ 0.05 ‘.’ 0.1 ‘ ’ 1

Log-likelihood: 23.61158
 
Covariance parameters Matern function (kappa=0.5) 
             Estimate StdErr
log(sigma^2)  -1.6873 0.2964
log(phi)       4.5827 0.4412
log(tau^2)    -1.7643 0.7568

Legend: 
sigma^2 = variance of the Gaussian process 
phi = scale of the spatial correlation 
tau^2 = variance of the nugget effect 
> 
\end{verbatim}


\begin{figure}
    \centering
    \includegraphics[width=0.75\linewidth]{plots/geo_predictive_points.png}
    \caption{Prediction Data Points}
    \label{fig:pred_poinrs}
\end{figure}

\begin{figure}
    \centering
    \includegraphics[width=0.75\linewidth]{plots/VO_Index_Estimates.png}
    \caption{VO Index Estimates (Mean)}
    \label{fig:placeholder}
\end{figure}

\begin{figure}
    \centering
    \includegraphics[width=0.75\linewidth]{plots/VO_Risk.png}
    \caption{High and Low Risk based On q1 \& q3}
    \label{fig:vo_risk}
\end{figure}

\begin{figure}
    \centering
    \includegraphics[width=0.75\linewidth]{plots/pred_elevation.png}
    \caption{Predicted Elevation}
    \label{fig:pred_elevation}
\end{figure}

\clearpage



\printbibliography



\newpage
\section{Appendix}
\begin{lstlisting}[language=R]
#Homework 1: SPEP
#########QUESTION 1 ####################
# load libraries
library(dplyr)
library(readxl)
library(tidyr) 
library(sf)
library(spdep)
library(tmap)
library(readr) 
library(stringr)
library(purrr)
library(broom)
library(INLA)



\end{lstlisting}





\end{document}
