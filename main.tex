 \documentclass[12pt]{article}
\renewcommand{\baselinestretch}{1.1}

%%% Add packages here
\usepackage{minted}
\usepackage{listings}
\usepackage{xcolor} % for syntax colors
\usepackage{graphicx} 
%\usepackage{parskip}
%\usepackage[table]{xcolor}
\usepackage{amsmath}
%\usepackage{float}
\usepackage{placeins} % prevent float
\usepackage{geometry}
\geometry{legalpaper, portrait, margin=3cm}
\usepackage{makecell}
\usepackage{multirow}
\usepackage{array}
\usepackage{amsfonts}
\usepackage{amsthm}
\usepackage{amssymb}
\usepackage{latexsym}
\usepackage{color}
\usepackage{verbatim}
\usepackage{fancyhdr}
\usepackage{fancybox}
\usepackage{listings}
\usepackage{threeparttable} % for table notes
\usepackage{hologo} % for TeX engine logos
\usepackage{booktabs} % for nice tables
\usepackage{longtable} % for longer tables
\usepackage[table,xcdraw]{xcolor}  % for color in tables
\usepackage{appendix} % for appendix
\usepackage{biblatex}
\addbibresource{references.bib}

\usepackage{amsmath} % for pseudo code
\usepackage{algorithm}
\usepackage{algpseudocode}
\usepackage{subcaption} % For subfigures
\usepackage{threeparttable} % for table notes

\definecolor{codegreen}{rgb}{0,0.6,0}
\definecolor{codegray}{rgb}{0.5,0.5,0.5}
\definecolor{codepurple}{rgb}{0.58,0,0.82}
\definecolor{backcolour}{rgb}{0.95,0.95,0.92}

\lstdefinestyle{mystyle}{
    backgroundcolor=\color{backcolour},   
    commentstyle=\color{codegreen},
    keywordstyle=\color{magenta},
    numberstyle=\tiny\color{codegray},
    stringstyle=\color{codepurple},
    basicstyle=\ttfamily\footnotesize,
    breakatwhitespace=false,         
    breaklines=true,                 
    captionpos=b,                    
    keepspaces=true,                 
    numbers=left,                    
    numbersep=5pt,                  
    showspaces=false,                
    showstringspaces=false,
    showtabs=false,                  
    tabsize=2
}

\lstset{style=mystyle}

%%% Margins
\addtolength{\oddsidemargin}{-.50in}
\addtolength{\evensidemargin}{-.50in}
\addtolength{\textwidth}{1.0in}
\addtolength{\topmargin}{-.40in}
\addtolength{\textheight}{0.80in}

%%% Header
\pagestyle{fancy}
\chead{\groupname} 
\rhead{}
\lhead{}
\cfoot{\thepage}
\renewcommand{\headrulewidth}{0.4pt}

%%%%%%%%%%%%%%%%%%%%%%%%%%%%%%%%%%%%%
\newcommand{\groupname}{Group 4 - Geostatistics}
\setcounter{secnumdepth}{0}
\begin{document}
\clearpage\thispagestyle{empty}

\begin{titlepage}

\begin{center}
\vspace*{1in}
	% title
	\centering\textbf{\huge{Spatial Epidemiology\\[3cm] }}
	% details
	\Large{\textbf{
	Master of Statistics and Data Science} \\
	Hasselt University\\	
        2024-2025 \\
	}


\vspace*{3cm}
\textbf{\large{Group 4 :}}\\
Winnie Kulei (2469362) \\
Edward Otieno Owinoh (2365191) \\
Loise Kanini (2262458)



\vspace*{2cm}


\vspace*{1.5cm}
\textbf{\large{Lecturers:}}\\
Prof. Dr. Christel Faes \\
Prof. Dr. Thomas Neyens \\

\vspace*{2\baselineskip}
\today

\begin{figure}[b]
   \centering
   \includegraphics[width=8cm]{UHasselt_logog.png}
   \label{fig:Uhasselt}
\end{figure}

\end{center}

\end{titlepage}


%%%%%%%%%%%%%%%%%%%%%%%%%%%%%%%%%%%%%%%%%%%%%%%%%%
\newpage
\section{Introduction}
\section*{Background}

Dengue is a viral infection transmitted by multiple mosquito species. One way to quantify the local
risk of dengue transmission is through the \textit{Vector Occurrence} (VO) index, a continuous measure
of the joint abundance of these mosquito species at a given location. The VO index is constructed
from mixed information obtained through both light traps and sentinel traps; thus, a larger VO value
indicates higher combined mosquito abundance.

This project investigates the spatial variation of the VO index across Cambodia. Because mosquito
distribution and abundance are affected by biophysical conditions, the VO index is expected to vary
with environmental factors.

\bigskip

\noindent
The dataset contains the following variables for each location $i$:
\begin{center}
\begin{tabular}{ll}
\hline
\textbf{Variable} & \textbf{Description} \\
\hline
VO\_i            & Vector Occurrence index at location $i$ \\[2pt]
elevation\_i     & Elevation at location $i$ (meters above sea level; a proxy for environmental variables) \\[2pt]
X\_i             & Longitude of location $i$ \\[2pt]
Y\_i             & Latitude of location $i$ \\
\hline
\end{tabular}
\end{center}

\section*{Assignment Tasks}

For this assignment, you are expected to:

\begin{enumerate}
    \item Check whether VO showcases spatial correlation.
    \item Check whether your analysis suggests that there are unobserved explanatory variables (possibly at a very small spatial scale) that are associated with variation in VO, other than elevation..
    \item Make a map of model-based predictions of the mean VO in Cambodia, as a function of elevation(if important) and other latent variables (if important).
    \item High-risk areas are those with a VO of 6 or larger. In their quest to understand which part of the country is considered high-risk, the Cambodian government wants to fund a campaign to collect data on VO at 10 additional locations. Choose a batch of 10 new locations, using an adaptive sampling design based on the currently available data.
\end{enumerate}


Summary statistics

% latex table generated in R 4.5.2 by xtable 1.8-4 package
% Mon Dec  8 12:47:46 2025
\begin{table}[ht]
\centering
\begin{tabular}{rrrrrrr}
  \hline
 & Min. & 1st Qu. & Median & Mean & 3rd Qu. & Max. \\ 
  \hline
VO & 0.00 & 5.60 & 6.11 & 5.92 & 6.60 & 8.13 \\ 
  elevation & -10.00 & 18.50 & 72.00 & 132.08 & 130.00 & 1347.00 \\ 
   \hline
\end{tabular}
\end{table}
\begin{figure}[htbp]
    \centering
    \includegraphics[width=0.75\linewidth]{plots/eda_VO.png}
    \caption{VO }
    \label{fig:vo}
\end{figure}

\begin{figure}[htbp]
    \centering
    \includegraphics[width=0.75\linewidth]{plots/eda_elevation.png}
    \caption{Elevation }
    \label{fig:el}
\end{figure}

\begin{figure}[htbp]
    \centering
    \includegraphics[width=0.75\linewidth]{plots/eda_VO_vs_elevation.png}
    \caption{Elevation vs VO }
    \label{fig:el_vo}
\end{figure}
\begin{verbatim}
    Call:
lm(formula = VO ~ elevation, data = VO_Cambodia_Updated)

Residuals:
     Min       1Q   Median       3Q      Max 
-1.29794 -0.41959 -0.07486  0.49279  1.64685 

Coefficients:
              Estimate Std. Error t value Pr(>|t|)    
(Intercept)  6.5549415  0.0488785  134.11   <2e-16 ***
elevation   -0.0047860  0.0002012  -23.79   <2e-16 ***
---
Signif. codes:  0 ‘***’ 0.001 ‘**’ 0.01 ‘*’ 0.05 ‘.’ 0.1 ‘ ’ 1

Residual standard error: 0.6001 on 212 degrees of freedom
Multiple R-squared:  0.7274,	Adjusted R-squared:  0.7262 
F-statistic: 565.8 on 1 and 212 DF,  p-value: < 2.2e-16
\end{verbatim}


\begin{figure}[htbp]
    \centering
    \includegraphics[width=0.75\linewidth]{plots/eda_VO_vs_elevation.png}
    \caption{Elevation vs VO }
    \label{fig:el_vo}
\end{figure}

\begin{figure}
    \centering
    \includegraphics[width=0.75\linewidth]{plots/spatial_correlation_diagnostic.png}
    \caption{Variogram}
    \label{fig:lm_variogram}
\end{figure}


\begin{verbatim}
    Least square fit to the empirical variogram 
sigma^2 = 0.1175444 (Variance of the Gaussian process) 
phi = 68.91269 (Scale of the spatial correlation) 
tau^2 = 0.06385818 (Variance of the nugget effect) 
\end{verbatim}

\begin{figure}
    \centering
    \includegraphics[width=0.9\linewidth]{plots/lm_diag_plots.png}
    \caption{Linear Model Diagnostic Plots}
    \label{fig:lm_Plots}
\end{figure}


Linear geostatistical Model
\begin{verbatim}
                   Estimate      StdErr z.value   p.value    
(Intercept)  6.54123347  0.10395944  62.921 < 2.2e-16 ***
elevation   -0.00468364  0.00022119 -21.174 < 2.2e-16 ***
---
Signif. codes:  0 ‘***’ 0.001 ‘**’ 0.01 ‘*’ 0.05 ‘.’ 0.1 ‘ ’ 1

Log-likelihood: 23.61158
 
Covariance parameters Matern function (kappa=0.5) 
             Estimate StdErr
log(sigma^2)  -1.6873 0.2964
log(phi)       4.5827 0.4412
log(tau^2)    -1.7643 0.7568
\end{verbatim}
\begin{figure}
    \centering
    \includegraphics[width=0.75\linewidth]{plots/geo_spatial_correlation_diagnostic.png}
    \caption{Geostatitical Variogram}
    \label{fig:geo_variogram}
\end{figure}



\begin{verbatim}
    Geostatistical linear model 
Call: 
linear.model.MLE(formula = VO ~ elevation, coords = ~utm_x + 
    utm_y, data = VO_Cambodia_Updated, ID.coords = NULL, kappa = 0.5, 
    fixed.rel.nugget = NULL, start.cov.pars = c(spatial_cor[["lse.variogram"]][["phi"]], 
        spatial_cor[["lse.variogram"]][["tau^2"]]/spatial_cor[["lse.variogram"]][["sigma^2"]]), 
    method = "nlminb")

               Estimate      StdErr z.value   p.value    
(Intercept)  6.54123347  0.10395944  62.921 < 2.2e-16 ***
elevation   -0.00468364  0.00022119 -21.174 < 2.2e-16 ***
---
Signif. codes:  0 ‘***’ 0.001 ‘**’ 0.01 ‘*’ 0.05 ‘.’ 0.1 ‘ ’ 1

Log-likelihood: 23.61158
 
Covariance parameters Matern function (kappa=0.5) 
             Estimate StdErr
log(sigma^2)  -1.6873 0.2964
log(phi)       4.5827 0.4412
log(tau^2)    -1.7643 0.7568

Legend: 
sigma^2 = variance of the Gaussian process 
phi = scale of the spatial correlation 
tau^2 = variance of the nugget effect 
> 
\end{verbatim}


\begin{figure}
    \centering
    \includegraphics[width=0.75\linewidth]{plots/geo_predictive_points.png}
    \caption{Prediction Data Points}
    \label{fig:pred_poinrs}
\end{figure}

\begin{figure}
    \centering
    \includegraphics[width=0.75\linewidth]{plots/VO_Index_Estimates.png}
    \caption{VO Index Estimates (Mean)}
    \label{fig:placeholder}
\end{figure}

\begin{figure}
    \centering
    \includegraphics[width=0.75\linewidth]{plots/VO_Risk.png}
    \caption{High and Low Risk based On q1 \& q3}
    \label{fig:vo_risk}
\end{figure}


\clearpage
\newpage
\section{Appendix}
\begin{lstlisting}[language=R]
#Homework 1: SPEP
#########QUESTION 1 ####################
# load libraries
library(dplyr)
library(readxl)
library(tidyr) 
library(sf)
library(spdep)
library(tmap)
library(readr) 
library(stringr)
library(purrr)
library(broom)
library(INLA)



\end{lstlisting}





\end{document}
